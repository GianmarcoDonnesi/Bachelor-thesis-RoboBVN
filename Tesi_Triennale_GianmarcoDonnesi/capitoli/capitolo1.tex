 \begin{sloppypar}
\chapter{Introduzione}
\fontsize{12}{19}\selectfont{
I disturbi dello spettro autistico (\textit{DSA}) si riferiscono a disturbi dello sviluppo neurologico con esordio nei primi tre anni di vita e sono caratterizzati  da  difficoltà  di  comunicazione, di interazione sociale e da modelli limitati e ripetitivi nei comportamenti, negli interessi  e nelle attività di un individuo.
 L’autismo  rappresenta  una sindrome molto complessa, nella maggior parte dei casi di natura genetica e poiché  le sue manifestazioni sono molto  varie  e  possono differire notevolmente da soggetto a soggetto, è  più appropriato  parlare  di  “spettro  autistico”.\newline
Per diversi anni i \textit{DSA} sono stati considerati essenzialmente rari con la prevalenza minore di un bambino affetto su mille. Ad oggi, il \textit{World Health Organization} stima che la prevalenza sia di un bambino su cento e sostiene che sembra essere in aumento negli anni futuri.
Essi vengono diagnosticati quando il soggetto, a seguito di test standardizzati somministrati individualmente, su lettura, calcolo, o espressione scritta, ottenga risultati significativamente al di sotto delle soglie predeterminate che tengono conto dell'età, dell'istruzione e del livello di intelligenza.\newline
Come confermato a più riprese dagli esperti, la diagnosi e l’avvio conseguente di un intervento precoce può migliorare notevolmente la prognosi dei bambini con DSA e la qualità di vita delle loro famiglie.
Purtroppo, la varietà e la diversità delle manifestazioni legate ai DSA, la limitata conoscenza degli stessi da parte delle famiglie, il numero insufficiente di centri specializzati e di professionisti esperti disponibili, concorrono a causare ritardi nella diagnosi e quindi nella cura.\newline
Le nuove tecnologie informatiche rispondono alla crescente esigenza di individuare le migliori pratiche per lo screening e la diagnosi dei DSA già nelle prime fasi dello sviluppo dell'individuo.
Esse, infatti, sono da tempo all’ attenzione del mondo medico, oltre che educativo e scientifico, per la valenza facilitante che possono avere nella progettazione di programmi abilitativi personalizzati/individualizzati \cite{Numero13}. 
In particolare, per i pazienti con DSA le tecnologie informatiche adeguatamente impiegate, costituiscono un valido supporto nei percorsi di intervento e di trattamento finalizzati al superamento dei deficit comunicativi e relazionali potenziando l’apprendimento, la comunicazione e la socializzazione.\newline
Esse consentono di usare un canale comunicativo prevalentemente visuo-spaziale, particolarmente adatto alle caratteristiche dei pazienti con disturbi dello spettro autistico. Inoltre, il linguaggio informatico, in quanto chiaro, strutturato e prevedibile, ha il grande pregio di non comportare inferenze emotive, incomprensibili per una persona con autismo.
Tra gli approcci più  innovativi vi è sicuramente quello relativo
alla possibilità  di utilizzare tecnologie assistive durante le sessioni di terapia. 
I rapidi progressi nell’area della robotica offrono oggi enormi possibilità di innovazione e di trattamento o addirittura di educazione per i pazienti affetti da DSA.
Fu lo scrittore ceco \textit{Karel Cˇapek}  che nel suo dramma fantascientifico 
teatrale \textit{R.U.R.} (\textit{I robot universali di Rossum}) nel 1920,  in  sostituzione  del  termine  “automa”,  ideò  il  termine “robot” che stava ad indicare una macchina meccanica con funzioni di lavoro. 
Successivamente, lo scrittore russo \textit{Isaac Asimov}  nel  1942  pubblicò   tre  leggi  ancor  oggi  geniali  nella loro intuizione. Si trattava delle tre leggi della robotica, nate per regolamentare e garantire il comportamento etico di una macchina dotata di intelligenza artificiale.\newline
All'epoca, tali leggi si limitavano ad apparire come l’acuta invenzione di uno scrittore di fantascienza. Oggi, invece, le ricerche scientifiche e tecnologiche hanno compiuto progressi tali da renderle quanto mai attuali; sono diventate un punto fermo della fantascienza e un fondamento dell’etica robotica.\newline
Da allora la robotica ha fatto dei progressi straordinari fino a  trasformare  gli  attuali  robot  in  entità autonome con capacità di movimento, apprendimento, comunicazione e adeguamento  all’ ambiente circostante,  fino  a  prospettare  la  possibilità  di  robot dotati di intelligenza artificiale (\textit{IA}).\newline
  La  robotica  di  assistenza  sociale (RAS) è quel campo della ricerca riguardante il modo in cui i robot assistono le persone attraverso l’interazione sociale. Essa costituisce un sottocampo dell’interazione uomo-robot (\textit{HRI}), volto allo sviluppo di interazioni efficienti con l’utente in contesti terapeutici  ed  educativi  per  aiutarlo  a  costruire  abilità  di  comportamento sociale \cite{Numero7}.\newline
L’utilizzo di interventi basati sulla RAS ha dimostrato in primo luogo il notevole potenziale nel trattamento dei DSA con l'aumento del coinvolgimento del paziente con difficoltà nella comunicazione. Inoltre, l'impiego di tali tecnologie ha messo a disposizione dei terapisti la possibilità di organizzare sessioni di trattamento più interattive \cite{Numero9}\cite{Numero10} con supporto negli aspetti valutativi (oggettivazione, raccolta dati), nonché la facoltà di delineare percorsi di cura riabilitativi innovativi e altamente personalizzati.
Il miglioramento ed il potenziamento dello sviluppo sociale, emotivo, cognitivo e sensoriale, sono i principali obiettivi terapeutici ed educativi per il trattamento e la cura dei DSA. Naturalmente, ognuna di queste aree richiede un approccio e tipologie di dispositivi specifici, per questo sono stati realizzati modelli di intervento distinti a seconda degli scopi della determinata competenza su cui lavorare.\newline
Nel contesto riabilitativo, le due categorie di robot che vengono maggiormente utilizzati sono quelli a scopo terapeutico e quelli a scopo assistivo. 
Nel presente lavoro di tesi, si è deciso di impiegare robot antropomorfi di tipo assistivo per favorire lo sviluppo di competenze emotive e sociali al fine di migliorare il benessere psicologico del paziente.\newline
Numerosi studi condotti su bambini affetti da disturbi dello spettro autistico hanno evidenziato risultati positivi in termini di interazione sociale.
I bambini hanno mostrato una maggiore interazione diadica e triadica,  uno sviluppo  migliore delle  capacità  di  comunicazione sociale e un’acquisizione di regole comportamentali più efficace.\newline 
Un robot umanoide è in grado di fornire gli stimoli sociali adeguati in modo molto simile a quello degli operatori umani, può essere riconosciuto facilmente come “compagno” e può essere programmato con applicazioni che sollecitano lo sviluppo di abilità interpersonali di cui i pazienti hanno bisogno: imitazione, contatto oculare e fisico, comunicazione verbale e non verbale. I robot possono attivare luci o musiche e compiono movimenti lenti e prevedibili al fine di fornire i rinforzi positivi di cui necessitano i pazienti.\newline
Allo stato attuale non esiste ancora un metodo esclusivo per confermare una diagnosi di autismo attraverso test di laboratorio, di neuroimaging o genetici; anche gli strumenti psicodiagnostici sono spesso difficili da impiegare per le difficoltà tipiche della sindrome. Rilevato che la diagnosi di autismo si basa in gran parte sulla valutazione clinica, l'avvalimento in fase diagnostica delle capacità del robot di rilevazione e registrazione delle risposte dell'individuo, consente di determinare con una certa accuratezza l'eventuale presenza di un paziente affetto da DSA.\newline
Le valutazioni sono favorite fornendo stimoli standardizzati per valutare i comportamenti caratteristici dell’autismo così da giungere a diagnosi integrate sempre più affidabili. 
La risposta del paziente agli stimoli del robot umanoide, al tono della sua voce, ai movimenti degli occhi, ai cambiamenti di espressione e al tipo di conversazione forma dati che vengono raccolti ed elaborati per una successiva analisi. Quest'ultima viene svolta da un lato in maniera totalmente automatizzata e dall'altro tramite una valutazione clinica di un esperto.\newline
Tutte le considerazioni e le ricerche sopra esposte hanno posto le basi per il presente lavoro di tesi, il quale ha avuto come obiettivo lo sviluppo di una applicazione RAS per la somministrazione di test per l’analisi delle funzioni cognitivo-linguistiche in individui affetti da disturbi dello spettro autistico \cite{Numero11}.
}
 \end{sloppypar}
 \afterpage{\blankpage}