 \begin{sloppypar}
\chapter{Conclusioni}
\fontsize{12}{19}\selectfont{
Negli ultimi anni la robotica sociale assistita (RSA) è stata oggetto di studio scientifico e di applicazione clinica nel trattamento e nella diagnosi del Disturbo dello Spettro Autistico (DSA).
Gli studi in questo settore hanno dimostrato che  l’uso  della  RSA  può  migliorare  l’attenzione  e  il  coinvolgimento dei pazienti affetti da disturbi dello spettro autistico, promuovere l’interazione sociale e comunicativa e supportare i professionisti del settore sanitario.\newline
In questo lavoro  di  tesi è  stata  presentata e descritta  un'applicazione  sviluppata per la somministrazione attraverso il robot umanoide \textit{Pepper - SoftBank Robotics}\texttrademark \space di test per l’analisi delle funzioni cognitivo-linguistiche in individui affetti da disturbi dello spettro autistico.\newline
L’applicazione sviluppata si è dimostrata efficace nel facilitare la somministrazione dei test e la raccolta dei risultati ottenuti.\newline
Essa consente un certo grado di standardizzazione delle procedure e un’analisi più veloce e dettagliata dei dati, generalmente collezionati solo in forma cartacea.\newline 
Ci si augura che gli sviluppi futuri previsti siano in grado nel breve tempo di rendere l’applicazione ancora più efficiente al fine di una valutazione sempre più accurata delle funzioni cognitive-linguistiche dei pazienti, facilitando la  gestione e l’elaborazione di informazioni utili alla diagnosi, alla pianificazione delle terapie e al monitoraggio dei loro progressi nel tempo.
}
 \end{sloppypar}
\afterpage{\blankpage}